\documentclass{article}
\usepackage[utf8]{inputenc}
\usepackage[spanish]{babel}
\usepackage{listings}
\usepackage{graphicx}
\graphicspath{ {images/} }
\usepackage{cite}
\usepackage{pstricks,pst-node}
\usepackage{schemata}

\begin{document}

\begin{titlepage}
    \begin{center}
        \vspace*{1cm}
            
        \Huge
        \textbf{Parcial II}
            
        \vspace{0.5cm}
        \LARGE
        Subtítulo
            
        \vspace{1.5cm}
            
        \textbf{Juan Diego Arias Toro}
            
        \vfill
            
        \vspace{0.8cm}
            
        \Large
        Despartamento de Ingeniería Electrónica y Telecomunicaciones\\
        Universidad de Antioquia\\
        Medellín\\
        Septiembre de 2021
            
    \end{center}
\end{titlepage}

\tableofcontents
\newpage
\section{Análisis del problema y consideraciones para la alternativa de solución propuesta.}\label{intro}
La forma de abordar el problema será primero desarrollar en programa en C++ (en el entorno de desarrollo de QT), un programa para que lea una imagen y escriba una secuencia de matrices de RGB, en donde mostrara los valores de cada color correspondiente a un valor, en rojo (R) uno en verde (G) y uno en azul (B).

Posteriormente se desarrollará un circuito en TinkerCad con un arduino en para colocar una entrada seria, en la que se ingresara la matriz posteriormente mencionada, con la ayuda una tira de LED inteligente.

Una de las clases implementadas en el algoritmo será MainWindow, para tener una parte donde subir la imagen a analizar, y posteriormente para entregar las matrices R, G y B; en este orden poder luego copiarlas a la entrada serial del arduino. 

\section{Esquema de tareas} \label{contenido}
\begin{center}
\begin{minipage}[c]{1\textwidth}
\schema{\schemabox{Parcial 2}}{
        	\schemabox{
        		\schema{\schemabox{$\bullet$ Planeacion}}
        	{
	        				\schemabox{$\bullet$ Investigracion}
    		    				\schemabox{$\bullet$ Revisar Clases}
        					\schemabox{$\bullet$ Solucionar inquietudes}
        		}
    		}
    		\schemabox{
    			\schema{\schemabox{$\bullet$ Ejecucion}}{
    						\schemabox{$\bullet$ DeSarrollo del circuito}
    						\schemabox{
    							\schema{\schemabox{$\bullet$ Desarrollo de los codigos}}{\schemabox{--Codigo c++ (En QT) }\schemabox{--Desarrollo del codigo de arduino}}}
    		}}
}
\end{minipage}
\end{center}
\section{Pseudocódigo} \label{contenido}
\subsection{C++}
\begin{lstlisting}
Algoritmo Convertir imagen a Matrices(RGB)

Variables a,b,c,d....z
Lectura ( Dirección de la imagen descargada )
Mientras seccion_recorrida < numero_total_de_pixeles Hacer
	Escribir numero_rojo_en_la_seccion_actual
	Escribir numero_verde_en_el_seccion_actual 
	Escribir numero_azul_en_el_seccion_actual
FinMientras

Mostrar
matriz_roja_completa
matriz_verde_completa
matriz_azul_completa

Fin
\end{lstlisting}
\subsection{Arduino}
\begin{lstlisting}
Algoritmo Integrar las matriz RGB al Arduino y la tira de Led

Variables a,b,c,d....z
Lectura ( Matriz con los valores del color rojo )
Lectura ( Matriz con los valores del color verde )
Lectura ( Matriz con los valores del color azul )
Mientras i < numero_total_leds Entonces
	i <- i+1
Mientras j < numero_total_leds Entonces
	j <- j+1
		Salida valor_rojo_[i][j]
	Mientras i < numero_total_leds Entonces
	i <- i+1
Mientras j < numero_total_leds Entonces
	j <- j+1
		Salida valor_verde_[i][j]
Mientras i < numero_total_leds Entonces
	i <- i+1
Mientras j < numero_total_leds Entonces
	j <- j+1
		Salida valor_azul_[i][j]
Entrega (Entrega los valor del rojo, el verde y el azul, 
por los puertos del arduino )

Fin
\end{lstlisting}
\section{Consideraciones} \label{contenido}
$\bullet$La dificultad debido a la ausencia de librerías aumenta mucho el manejo de las imagenes par determinar su matriz, por eso mismo se debe ser cuidadoso al momento de programar en código de C++ en donde tendremos que saber que esos sectores que se convertirán en imágenes en la matriz LED no pueden contener mucha información.

$\bullet$Las imágenes que se carguen en el código de C++ deben ser .png

$\bullet$Tener cuidado al utilizar colores blancos o cercanos al blanco, pues por la cantidad de energía que requieren puede resultar en una sobrecarga para el arduino.

$\bullet$No se podrá cargar más de una imagen en el Arduino pues si capacidad de procesamiento es limitada y no soportará más de una imagen.

\end{document}